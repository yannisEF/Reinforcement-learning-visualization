%%%%%%%%%%%%%%%%%%%%%%%%%%%%%%%%%%%%%%%%%%%
%%% DOCUMENT PREAMBLE %%%
\documentclass[12pt]{article}
\usepackage[T1]{fontenc}
%\usepackage{natbib}
\usepackage{url}
\usepackage[utf8x]{inputenc}
\usepackage{amsmath}
\usepackage{graphicx}
\graphicspath{{images/}}
\usepackage{parskip}
\usepackage{fancyhdr}
\usepackage{vmargin}
\usepackage{mathtools}
\usepackage{enumitem}
\setmarginsrb{2 cm}{2.5 cm}{2 cm}{2.5 cm}{1 cm}{1.5 cm}{1 cm}{1.5 cm}
\usepackage{subfigure}
\usepackage{blindtext}
\usepackage{tcolorbox}
\tcbuselibrary{minted,breakable,xparse,skins}

\definecolor{bg}{gray}{0.95}
\DeclareTCBListing{mintedbox}{O{}m!O{}}{%
  breakable=true,
  listing engine=minted,
  listing only,
  minted language=#2,
  minted style=default,
  minted options={%
    linenos,
    gobble=0,
    breaklines=true,
    breakafter=,,
    fontsize=\small,
    numbersep=8pt,
    #1},
  boxsep=0pt,
  left skip=0pt,
  right skip=0pt,
  left=25pt,
  right=0pt,
  top=3pt,
  bottom=3pt,
  arc=5pt,
  leftrule=0pt,
  rightrule=0pt,
  bottomrule=2pt,
  toprule=2pt,
  colback=bg,
  colframe=orange!70,
  enhanced,
  overlay={%
    \begin{tcbclipinterior}
    \fill[orange!20!white] (frame.south west) rectangle ([xshift=20pt]frame.north west);
    \end{tcbclipinterior}},
  #3}

\title{Rapport de projet}
% Title
\author{
Yannis Elrharbi-Fleury \\
Yuan Fangzheng
}						
% Author
\date{25/05/2021}
% Date

\setcounter{secnumdepth}{4}

\makeatletter
\let\thetitle\@title
\let\theauthor\@author
\let\thedate\@date
\makeatother

\pagestyle{fancy}
\rhead{\thedate}
\lhead{\thetitle}
\cfoot{\thepage}
%%%%%%%%%%%%%%%%%%%%%%%%%%%%%%%%%%%%%%%%%%%%
\begin{document}
\bibliographystyle{IEEEtran}
%%%%%%%%%%%%%%%%%%%%%%%%%%%%%%%%%%%%%%%%%%%%%%%%%%%%%%%%%%%%%%%%%%%%%%%%%%%%%%%%%%%%%%%%%

\begin{titlepage}
	\centering
    \vspace*{0.5 cm}
   \includegraphics[scale = 0.075]{Images/logo_SU.jpeg}\\[1.0 cm]	% University Logo
\begin{center}    \textsc{\Large   Sorbonne Université}\\[2.0 cm]	\end{center}% University Name
	\textsc{\Large Résolution de problème : problème d'ordonnancement}\\[0.5 cm]				% Course Code
	\rule{\linewidth}{0.2 mm} \\[0.4 cm]
	{ \huge \bfseries \thetitle}\\
	\rule{\linewidth}{0.2 mm} \\[1.5 cm]
	
	\begin{minipage}{0.4\textwidth}
		\begin{flushleft} \large
		\emph{Encadrant :}\\
		Evripidis Bampis\\
		\textbf{\Large }
			\end{flushleft}
			\end{minipage}~
			\begin{minipage}{0.4\textwidth}
            
			\begin{flushright} \large
			\emph{Étudiants :}\\
			\theauthor
		\end{flushright}
           
	\end{minipage}\\[2 cm]
	
\end{titlepage}

%%%%%%%%%%%%%%%%%%%%%%%%%%%%%%%%%%%%%%%%%%%%%%%%%%%%%%%%%%%%%%%%%%%%%%%%%%%%%%%%%%%%%%%%%
\newpage																		
\renewcommand*\contentsname{Table des Matières}
\tableofcontents 

\newpage
%%%%%%%%%%%%%%%%%%%%%%%%%%%%%%%%%%%%%%%%%%%%%%%%%%%%%%%%%%%%%%%%%%%%%%%%%%%%%%%%%%%%%%%%%
\setlength{\parindent}{2ex}

\section{Introduction}
Ce projet porte sur l'étude de solutions au problème d'ordonnancement. \par

Étant donné $N$ tâches et $1$ machine, il s'agit de trouver un ordonnancement de ces tâches minimisant la somme de leur temps de complétude (minimiser le temps d'attente total). \par

La machine ne connaît pas forcément leur durée d'exécution réelle. \par

Dans ce rapport, nous étudions et mesurons la qualité de plusieurs solutions en fonction de l'erreur de prédiction. \\

Le code est écrit en Python et sera fourni en annexe.

\newpage
\section{Structure du programme}

Nous avons naturellement opté pour une approche orientée objet du problème.

\subsection{Les distibutions}
La classe \emph{Distribution} représente un ensemble de distributions de probabilité et leurs paramètres. \\

Lors de son instanciation, elle prend en argument des fonctions permettant de générer un tuple de valeurs. \\

La méthode \emph{sample} renvoie un tuple contenant :
\begin{itemize}
	\item une durée réelle
	\item une durée erreur de prédiction
	\item un instant d'arrivée
\end{itemize}

%% DECRIRE CHOIX DE DISTRIBUTION

\subsection{Les tâches}
La classe \emph{Task} représente une tâche, dont les attributs sont générés à partir d'un objet de type \emph{Distribution}. \\

Elle possède notamment comme attributs :
\begin{itemize}
	\item un ensemble de durée : durée réelle, durée prédite (générées à partir de \emph{Distribution})
	\item un état : \emph{paused, running, finished, not available}
	\item un curseur \emph{currentStep} permettant d'avancer dans l'exécution de la tâche
	\item un numéro d'identification 
\end{itemize}

Une tâche possède trois méthodes :
\begin{itemize}
	\item \emph{hasFinished} : renvoie si la tâche est achevée ou non
	\item \emph{forward} : exécute la tâche d'un pas de temps, renvoie une exécution de \emph{hasFinished}
	\item \emph{restart} : réinitialise la tâche à son état initial
\end{itemize}

\subsection{Les machines}
Notre idée était de créer une classe \emph{Machine} représentant une machine capable de travailler sur un ensemble de tâches. Les différents algorithmes que nous présenterons dans la partie suivante en héritent. \\

Chaque machine possède notamment comme attributs :
\begin{itemize}
	\item des dictionnaires de tâches à différents états : \emph{allTasks, workingTasks, pausedTasks, finishedTasks}
	\item une vitesse d'exécution
	\item une horloge donnant le temps de la machine
	\item une clée d'affichage (une fonction lmabda) permettant de trier les tâches de la machine lors de son affichage
\end{itemize}

Ainsi, \emph{Machine} possède plusieurs méthodes concernant ses tâches : ajouter ou supprimer des tâches; démarrer, mettre en pause ou terminer une tâche. \\

Elle possède aussi des méthodes permettant de les traîter :
\begin{itemize}
	\item une méthode de travail \emph{work} faisant travailler les tâches sur un pas de temps de la machine
	\item une méthode abstraite de traitement \emph{run} traîtant les tâches avant chaque étape de travail, elle permet d'introduire l'algorithme de la machine
	\item une méthode de démarrage \emph{boot} démarrant la machine et la faisant tourner jusqu'à ce que toutes ses tâches soient terminées
\end{itemize}

\subsection{Les machines parallèles}
La classe \emph{Parallel} fonctionne de façon analogue à \emph{Machine}, à ceci près qu'elle ne fournit pas le travail aux tâches elle même. \\

Le travail sur les tâches est effectué par deux machines (\emph{Prediction} et \emph{Round-Robin}) que possède \emph{Parallel}, les exécutant avec une vitesse $\lambda$ et $1 - \lambda$.

\newpage
\section{Implémentation des algorithmes}

Grâce à notre refléxion claire sur les structures de données, nous avons pu facilement implémenter les aglorithmes demandés. \\

Dans cette partie, nous nous contenterons de présenter le fonctionnement de ceux-ci. Nous étudierons leurs performances dans la prochaine section. \\

Les algorithmes suivant héritent de \emph{Machine} ou \emph{Parallel} et ne font que surchager la méthode \emph{run}. \\

\subsection{Algorithme optimal : Shortest processing time}

L'agorithme \emph{SPT} est un algorithme optimal pour ce problème. \\

Il connaît la durée réelles des tâches et les exécute de la plus courte à la plus grande. \\ 

\begin{mintedbox}{python}
    def run(self, step):
        if len(self.workingTasks) == 0:
            nextTask = sorted(list(self.pausedTasks.values()), key=lambda x:x.realLength)[0]
            self.startTask(nextTask)
        return self.work(step)
\end{mintedbox}

Sa méthode \emph{run} est claire : si aucune tâche n'est en cours d'exécution, alors la machine exécute la plus courte.

\subsection{Exécution par prédiction}

Pour l'algorithme \emph{Prediction}, la machine n'a pas accès aux durées réelles des tâches. Elle ne connait que leur durée prédite, avec par conséquent une certaine erreur. \\

\begin{mintedbox}{python}
    def run(self, step):
        if len(self.workingTasks) == 0:
            nextTask = sorted(list(self.pausedTasks.values()), key=lambda x:x.predLength)[0]
            self.startTask(nextTask)
        return self.work(step)
\end{mintedbox}

La méthode est quasiment identique à l'algorithme précédent, à ceci près que les tâches sont triées en fonction de leur durée prédite.

\subsection{Round-Robin}

L'algorithme \emph{Round-Robin} partage son travail de façon égale entre les tâches. \\

Cet algorithme est utile dans le cas où les prédictions sont mauvaises, car il possède un rapport de compétitivité $\max_i {\frac {A(I)} {OPT(I)}}$ de $2$. \\

Notre implémentation se déroule en deux phases : l'initialisation (toutes les tâches démarrent) et l'exécution (\emph{run}). \\

\begin{mintedbox}{python}
    def _initRun(self):
        for task in self.allTasks.values():
            self.startTask(task)
            
    def run(self, step):
        if self.currentTime == 0:
            self._initRun()

        self.speed = self.initSpeed / len(self.workingTasks)
        return self.work(step)
\end{mintedbox}

Il suffit de changer la vitesse de la machine en fonction du nombre de tâches restantes à chaque pas de temps.

\subsection{Exécution parallèle}

Comme vu précédemment, \emph{Parallel} exécute \emph{Predicition} et \emph{Round-Robin} aux vitesses $\lambda$ et $1 - \lambda$. \\

Ainsi, lors de l'initialisation de la machine, on définit ses sous-machines :
\begin{mintedbox}{python}
        self.speed = speed
        self.prediction = Prediction(speed * lmb, key)
        self.roundRobin = RoundRobin(speed * (1-lmb), key)
\end{mintedbox}

Les deux machines possèdent les même tâches en référence, et travaillent ensemble à leur avancement. \\

La méthode \emph{run} prend la forme suivante :
\begin{mintedbox}{python}
    def run(self, step):
        self.currentStep += step

        self.finishTasks()

        if not bool(self):
            self.prediction.run(step)
            self.roundRobin.run(step)

        return bool(self)
        
    def __bool__(self):
        return bool(self.prediction) or bool(self.roundRobin)
\end{mintedbox}

Avec les méthodes \emph{bool} renvoyant si les machines ont fini leur exécution ou non.

\subsection{Exécution parallèle dynamique}
%% A FINIR DE REDIGER

Nous avons eu l'idée d'expérimenter une version de machine parallèle au $\lambda$ dynamique. \\

Lorsqu'une tâche s'achève, on regarde à intervalle régulier la différence entre le temps d'exécution prédit, et le temps réellement effectué. \\

On utilise la formule suivante : $\lambda' = e^{-\epsilon_m}$ avec $\epsilon_m$ la moyenne de l'erreur observée. \\

\begin{mintedbox}{python}
    def updateLmb(self):
        if len(self.finishedTasks) != 0 and len(self.finishedTasks) % 5 == 0:
            meanDiff = np.mean([abs(task.currentStep - task.predLength) for task in self.finishedTasks.values()])
            newLmb = np.exp(-meanDiff*self.coef)
    
            self.prediction.speed = self.speed * newLmb
            self.roundRobin.initSpeed = self.speed * (1-newLmb)
            self.historyLmb.append(newLmb)

    def finishTasks(self):
        super().finishTasks()
        self.updateLmb()
\end{mintedbox}

On s'attend alors à observer un déplacement exponentiel du premier ordre des performances de la machine vers celles de \emp{Round-Robin}. \\

Il s'agit de rajouter un coefficient $\alpha$ dans l'exponentielle pour adapter la vitesse de convergence de $\lambda$. \\

\newpage

\section{Test du bon fonctionnement des modèles}
    \subsection{Modèle Prédiction}
    \begin{itemize}
        \item[1]\includegraphics[width=15cm]{trace_0_pred.png} \\
        À l'instant 0 toutes les tâches ne sont pas disponibles.
        
        \item[2]\includegraphics[width=15cm]{trace_1_pred.png}
        
        À l'instant 6, la tâche 1 et 2 devienne disponibles, Prédiction réussi à sélectionner la tâche qui a la moins durée prédite.
        
        \item[3] \includegraphics[width=15cm]{trace_2_pred.png} 
        
        À l'instant 17, la tâche 1 est terminée. Ensuite la tâche 0 a été sélectionnée.
        
        \item[4]  \includegraphics[width=15cm]{trace_3_pred.png}
        
        Toutes les tâches sont terminées à l'instant 42.
    
    \end{itemize}
    
    \subsection{RR}
    \begin{itemize}
        \item[1]\includegraphics[width=15cm]{trace_0_rr.png}
        
        À l'instant 6, la tâche 7 a été sélectionné et allouée une vitesse de 1.
        
        \item[2]\includegraphics[width=15cm]{trace_1_rr.png}
        
        À l'instant 9, toutes les tâches sont exécutées avec une vitesse 0.2.
        
        \item[3]\includegraphics[width=15cm]{trace_2_rr.png}
        
        À l'instant 37, la tâche 7 est terminée. Les tâches restantes ont une vitesse 0.25.
        
        \item[3]\includegraphics[width=15cm]{trace_3_rr.png}
        
        À l'instant 76 toutes les tâches sont terminées.
        
    \end{itemize}

    \subsection{SPT}
    
    \begin{itemize}
        \item[1]\includegraphics[width=15cm]{trace_0_spt.png}
        
        À l'instant 6, la tâche 1 a été sélectionnée.
        
    \end{itemize}
    
    \subsection{Parallèle}
    
    \begin{itemize}
        \item[1]\includegraphics[width=15cm]{trace_0_para.png}
        
        À l'instant 1, la tâche 1 a été sélectionnée par Prédiction.
        
        Donc à l'instant 2, la tâche 1 a été exécutée 0.5*0.5 + 0.5 = 0.75, alors que la tâche 2 a 0.5*0.5 = 0.25 exécutée seulement par RR.
        
        \item[2]\includegraphics[width=15cm]{trace_1_para.png}
        
        À l'instant 4, la tâche 0 est disponible. Le temps d'exécution par RR est 0.5*0.33 = 0.165.
        
        Donc à l'instant 5, la tâche 1 a 8.75-(0.165+0.5)=8.085 et les autres ont seulement un temps d'exécution 0.165.
        
        \item[3]\includegraphics[width=15cm]{trace_3_para.png}
        
        La tâche 2 a été allouée un temps d'exécution=0.5+0.5*1=1.
        
        Le processus exécute jusqu'à la fin.
        
        
    \end{itemize}    
    
\newpage
\section{Résultats expérimentaux}

    Pour effectuer les test, nous utilisons les distributions de probabilités suivantes:
    \begin{itemize}
        \item Distribution de Pareto pour la durée de tâche
        \item Distribution normale pour le bruit de la prédiction
        \item Distribution uniforme pour les dates d'arrivée des tâches. Ça indique que les tâches arrivent au fur et à mesure.
    \end{itemize}

    \subsection{Statistiques des tâches}
    
    \begin{figure}
        \centering
        \subfigure[]{\includegraphics[width=0.4\textwidth]{Histogram of Arrival Time.png}} 
        \subfigure[]{\includegraphics[width=0.4\textwidth]{Histogram of Error.png}} 
        \subfigure[]{\includegraphics[width=0.4\textwidth]{Histogram of Real Length.png}}
        \label{fig:foobar}
    \end{figure}

    

    De plus, pour chaque algorithme nous tirons 100 tâches et observons les performances moyennes sur 100 exécutions différentes.

    \subsection{Introduction des dates d'arrivée}
    
    \includegraphics{fig_pred_noice.png}
    
    Nous pouvez visualiser très clairement que le temps de complétion augmente si la prédiction devient de plus en plus bruitée.
    
        
    \includegraphics[width=15cm]{fig_3-1.png}
    
    
    \begin{itemize}
        \item[1] SPT devrait avoir de meilleures performances que tous les modèles.
        \item[2] Si le bruit des durées des tâches augmente, alors la performance de la Prédiction devient de plus en plus mauvaise. Parallel devrait rester constante, RR devrait rester constante aussi.
        \item[3] RR a un rapport de compétitivité de 2 * SPT.
        \item[4] La machine Parallèle Auto-Lambda converge vers RR car moins la prédiction est bonne, plus on favorise RR.
        \item[5] Parallèle Avec Lambda=0.5 Fixé est un compromis entre Prediction et RR, sa performance décroît avec celle de Prediction.
  
    
    Les résultats expérimentaux sont cohérents avec nos hypothèses.
    \end{itemize}
    
    \includegraphics[width=15cm]{fig4-1.png}
    
    \subsection{Cadre classique}
    
    Dans ce cadre-là, il s'agit d'annuler la date d'arrivée. Ensuite tout reste le même. Nous avons constaté que le temps de complétion est plus grand. C'est parce que nous avons la date d'arrivée où la tâche ne peut pas être exécutée.
    
    \includegraphics[width=15cm]{fig3.png}
    
    \includegraphics[width=15cm]{fig4.png}
    
    
\end{document}
